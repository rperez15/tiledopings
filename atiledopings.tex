
\documentclass[review]{elsarticle}

\usepackage{lineno,hyperref}
\modulolinenumbers[5]

\journal{Nuclear Instruments and Methods B}

%%%%%%%%%%%%%%%%%%%%%%%
%% Elsevier bibliography styles
%%%%%%%%%%%%%%%%%%%%%%%
%% To change the style, put a % in front of the second line of the current style and
%% remove the % from the second line of the style you would like to use.
%%%%%%%%%%%%%%%%%%%%%%%

%% Numbered
%\bibliographystyle{model1-num-names}

%% Numbered without titles
%\bibliographystyle{model1a-num-names}

%% Harvard
%\bibliographystyle{model2-names.bst}\biboptions{authoryear}

%% Vancouver numbered
%\usepackage{numcompress}\bibliographystyle{model3-num-names}

%% Vancouver name/year
%\usepackage{numcompress}\bibliographystyle{model4-names}\biboptions{authoryear}

%% APA style
%\bibliographystyle{model5-names}\biboptions{authoryear}

%% AMA style
%\usepackage{numcompress}\bibliographystyle{model6-num-names}

%% `Elsevier LaTeX' style
\bibliographystyle{elsarticle-num}
%%%%%%%%%%%%%%%%%%%%%%%

\begin{document}

\begin{frontmatter}

\title{Performance of scintillator tiles with different doping concentrations after irradiation }


%% or include affiliations in footnotes:
\author[umd]{Geng-Yuan Jeng\corref{mycorrespondingauthor}}
\cortext[mycorrespondingauthor]{Corresponding author}
\ead{Geng-Yuan.Jeng@cern.ch}
\author[umd]{Alberto Belloni}
\author[umd]{Sarah C. Eno}
\author[baylor]{Kenichi Hatakeyama}
\author[princeton]{Christopher Tully}



\address[umd]{Dept. Physics, U. Maryland, College Park MD 30742 USA}
\address[eljen]{Eljen Technology, 1300 W. Broadway, Sweetwater, Tx 79556 USA}
\address[fnal]{Fermi National Accelerator Laboratory, Batavia, IL, USA}
\address[baylor]{Baylor University, Waco, Texas, USA}
\address[princeton]{Princeton University, Princeton, NJ, USA}

\begin{abstract}
The performance of plastic scintillator degrades when exposed to radiation. We present results on degradation of the light output of scintillator
tiles when irradiated by a $\rm{^{60}Co}$ source for a variety of concentrations of the primary and secondary dopant.  Tiles made from a blue scintillator with blue-to-green wavelength shifting fiber and for green scintillator with green-to-orange wavelength shifting fiber are presented.
\end{abstract}

\begin{keyword}
organic scintillator\sep radiation hardness \sep calorimetry
\end{keyword}

\end{frontmatter}

\linenumbers

\section{Introduction}
ampling calorimeters using plastic scintillator tiles with wave length shifting fibers, such as the CDF plug calorimeter \cite{Albrow20022524}, are popular due to\
 their low cost and ease of construction. 

\section{Conclusions}

\section{Acknowledgements}

This work was supported in part by U.S. Department of Energy Grant YYYYY.

\section*{References}

\bibliography{atiledopings}

\end{document}
