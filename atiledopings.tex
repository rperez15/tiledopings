
\documentclass[review]{elsarticle}

\usepackage{lineno,hyperref,color}
\modulolinenumbers[5]

\journal{Nuclear Instruments and Methods B}

%%%%%%%%%%%%%%%%%%%%%%%
%% Elsevier bibliography styles
%%%%%%%%%%%%%%%%%%%%%%%
%% To change the style, put a % in front of the second line of the current style and
%% remove the % from the second line of the style you would like to use.
%%%%%%%%%%%%%%%%%%%%%%%

%% Numbered
%\bibliographystyle{model1-num-names}

%% Numbered without titles
%\bibliographystyle{model1a-num-names}

%% Harvard
%\bibliographystyle{model2-names.bst}\biboptions{authoryear}

%% Vancouver numbered
%\usepackage{numcompress}\bibliographystyle{model3-num-names}

%% Vancouver name/year
%\usepackage{numcompress}\bibliographystyle{model4-names}\biboptions{authoryear}

%% APA style
%\bibliographystyle{model5-names}\biboptions{authoryear}

%% AMA style
%\usepackage{numcompress}\bibliographystyle{model6-num-names}

%% `Elsevier LaTeX' style
\bibliographystyle{elsarticle-num}
%%%%%%%%%%%%%%%%%%%%%%%

\begin{document}

\begin{frontmatter}

\title{Performance of scintillator tiles with different doping concentrations after irradiation }


%% or include affiliations in footnotes:
\author[umd]{Geng-Yuan Jeng\corref{mycorrespondingauthor}}
\cortext[mycorrespondingauthor]{Corresponding author}
\ead{Geng-Yuan.Jeng@cern.ch}
\author[umd]{Alberto Belloni}
\author[umd]{Sarah C. Eno}
\author[baylor]{Kenichi Hatakeyama}
\author[princeton]{Christopher Tully}
\author[umd]{Yao Yao}


\address[umd]{Dept. Physics, U. Maryland, College Park MD 30742 USA}
\address[eljen]{Eljen Technology, 1300 W. Broadway, Sweetwater, Tx 79556 USA}
\address[fnal]{Fermi National Accelerator Laboratory, Batavia, IL, USA}
\address[baylor]{Baylor University, Waco, Texas, USA}
\address[princeton]{Princeton University, Princeton, NJ, USA}

\begin{abstract}
The performance of plastic scintillator degrades when exposed to radiation. 
The degradation at low dose rates may be ameliorated by using scintillator
with higher dopant concentrations and with light output at longer wavelengths.
We present results on the degradation of light output of scintillator
tiles with embedded wavelength shifting  fibers  when irradiated by a $\rm{^{60}Co}$ source for a variety of concentrations of the primary and secondary dopant.  Tiles made from a blue scintillator with blue-to-green wavelength shifting fiber and for green scintillator with green-to-orange wavelength shifting fiber are presented.
\end{abstract}

\begin{keyword}
organic scintillator\sep radiation hardness \sep calorimetry
\end{keyword}

\end{frontmatter}

\linenumbers

\section{Introduction}
Sampling calorimeters using plastic scintillator tiles
with wavelength-shifting (WLS) fibers as the active element
have been part of hadron
collider experiments since the mid 1990's, when the CDF plug
calorimeter was constructed\cite{Aota1995557}.  Both the CMS
Barrel\cite{CMSHB} and Endcap\cite{HCALTDR1997} calorimeters use a similar design.
Prolonged exposure of plastic scintillator to
ionizing radiation, however, can result in damage:
light self-absorption (yellowing) increases and
the transfer efficiency of the initial excitation of the polymer to the
dopants combined with the probability of radiative decays for the dopants (``initial light output'') can lessen.  
During the running of the LHC from its commissioning in 2009
through 2012, the CMS
detector was exposed to an integrated luminosity of 25 ${\rm fb^{-1}}$.  Parts of the
CMS endcap calorimeter are estimated to have received doses of 0.1 to 0.2 Mrad\cite{ecfa2015}.
Studies of the radiation hardness of scintillator tiles
prior to installation in the detector,
using an electron linac and ${\rm ^{60}Co}$ sources,
indicated an expoential reduction in 
light output with accumulated dose, with a expoential constant of 
around 7 Mrad\cite{vasken,ByonWagner1993263}.  
However, although the dose received by the CMS tiles was
small compared to this number,
significant light loss was observed \cite{phaseiitdr}.


The effect of radiation on plastic scintillator is known to depend
both on dose and dose rate ~\cite{sauli,34504,Wick1991472,289295,173180,173178,Giokaris1993315}.  The increased self-absorption immediately after exposure is larger at high dose rate. However, after exposure, interactions 
with oxygen that diffuses into
the plastic decreases the initial damage, and the ``permenant'' damage after
a recovery time (typically a month) is usually indepenent of dose rate.  
While the permenant damage to the light self absorption may be
independent of dose rate, some studies indicate that the 
permenent damage to the initial light output depends
on dose rate\cite{Biagtan1996125} and that increasing the dopant concentration
can help alieviate this\cite{zorn3,sauli}.  The dose rates 
insitu at hadron collider experiments being much lower than those
used for reactor and linac tests may be part of the explanation
for the higher-than-expected damage to the CMS tiles.

In addition, many studies have shown that induced self-absorption is stronger 
at shorter wavelengths and thus scintillators that produce green light should
be more radiation resistant than the more common blue scintillators
\cite{Bross199135,sauli,Giokaris1993315}.  Dose rate effects may
therefore be smaller for such scintillators as well.

In this paper, we present measurements of ratio of the light output before and after irradiation
for tiles based on two different types of plastic scintillator manufactured by Eljen Technology, EJ-200 (a blue scintillator, 
similar to BC-408 from Bicron corporation) 
and 
EJ-260 (a green scintillator, similar to BC-428), before and after irradiation by a $\rm {^{60}Co}$ source for doses of 50, 30, 10, 4, and 2 Mrad at various dose rates and for different concentrations of the primary and secondary dopant.


\section{Tile design}
We tested two different tile designs.  Both used scintillator
with dimensions of 10 cm x 10 cm x 4 mm.  A blue-to-green multi-clad WLS fiber
from Kuraray (Y-11) with a diameter of 1mm was used for with the EJ-200.  A green-to-orange multi-clad WLS fiber from Kuraray Corporation 
(S-type O-2) with a dye concentration of 100 ppm and a diameter of 1 mm was used with the EJ-260.  Aluminum was sputtered onto one end of the fiber
to increase the light output.
A square ``${\rm \sigma}$''-shaped groove similar to that used for the 
CMS tiles with machined into the plastic, and the fiber was inserted
into the groove.  The tiles were wrapped with a tyvek covering, held together with tape.


\section{Results}
The tiles were irradiated at the University of Maryland
${\rm ^{60}Co}$ source.  The dose was measured using
{\color{red} I don't know.}  The light output was measured using cosmic rays.
Scintillator-based counters above and below the tile were used for triggers.
For the EJ-200 tile, the light output was measured using a Hamamatsu
R7600U-200-M4 photomultiplier tube.  As photomultiplier tubes have lower sensitivity for orange light, a {\color{red} sipmm of some sort} was used.

The fiber was connected
to the tube using optical glue.  Data was collected with a Tektronix MSO 5204 oscilloscope.  No attempt was made to select minimum ionizing (mip) muons.  The muons were thus
of low energy and produce more light than mips.


\section{Conclusions}

\section{Acknowledgements}
The authors would like to thank {\color{red} various people} at
the University of Maryland's Nuclear Reactor and Radiation
Facilities group for assistance
with the irradiations.
This work was supported in part by U.S. Department of Energy Grant DESC0010072.

\section*{References}

\bibliography{atiledopings}

\end{document}
